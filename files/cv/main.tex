\documentclass{clean_cv}

% Add a BibTeX-style file encoding all of your publications to include here. You can export this from Zotero. Only include
% publications you want to appear here!
\addbibresource{pub.bib}

\author{Kaiwen "Kevin" Dong}
\headlineposition{Ph.D student, University of Notre Dame}

\begin{document}

\maketitle
% In this section, you can use any of the FontAwesome icons. The commands \faCenter and \faCenterStyle have been defined to properly center the icons
% when using the default font settings.
%
% You can use any of the icons listed in the fontawesome5 package documentation (https://ctan.math.utah.edu/ctan/tex-archive/fonts/fontawesome5/doc/fontawesome5.pdf)
% If you need to specify a specific style (as is done here for the address card), you should use the two-argument \faCenterCycle command
\begin{center}
\begin{tabular}{lll}
    % & \faCenter{phone-alt} 541-754-301
    \faCenter{envelope} \href{mailto:kdong2@nd.edu}{kdong2@nd.edu}  & \faCenterStyle{regular}{address-card} 384 Nieuwland Science Hall, Notre Dame, IN 46556 USA & \faCenter{linkedin} \href{https://www.linkedin.com/in/kaiwen-dong-ab0634103/}{kaiwen-dong-ab0634103} \\
    \faCenter{orcid} \href{https://orcid.org/0000-0001-8244-9562}{0000-0001-8244-9562} & \faCenter{globe} \url{https://barcavin.github.io/} & \faCenter{github} \href{https://github.com/Barcavin}{Barcavin}\\
\end{tabular}
\end{center}

\vspace{-1.5em}

\section{Education}

% The datetabular environment takes one argument, which is the width of the left date column. As seen here:
%   9em is a good choice for "dual-date" formats (e.g. Sep 2015 - Nov 2019).
%   4em is a good choice for month/year dates (Sep 2014).
%   2em is a good choice for year-only dates (as seen in the publications)
\begin{datetabular}{9em}
% This is just a tabular environment, for the most part. The dateentry command has been defined for
% convienence. It takes two arguments, the first is the date and the second is whatever you wish placed to the right.
\dateentry{Aug 2021 -- Present}{
\textbf{University of Notre Dame}

\textit{Ph.D student in Computer Science, Teaching Assistant, Research Assistant longerr}
}

\dateentry{Aug 2016 -- May 2018}{
\textbf{University of Illinois at Urbana-Champaign}

\textit{Master of Science in Statistics}
}

\dateentry{Sep 2012 -- June 2016}{
\textbf{University of Illinois at Urbana-Champaign}

\textit{Bachelor of Science in Mathematics and Applied Mathematics}
}
\end{datetabular}

\section{Work Experience}
\begin{datetabular}{9em}
\dateentry{2021 - 2021}{
\textbf{University of Notre Dame}

\textit{Graduate Teaching Assistant}

\begin{itemize}
\item Data Structure(Fall 2021)
\end{itemize}\eatvspace
}
\dateentry{2018 - 2021}{
\textbf{Aunalytics}

\textit{Data Scientist}

\begin{itemize}
\item Leaded the development of the text-to-SQL product, which translates a natural language utterance to a machine-executable SQL query.
\end{itemize}\eatvspace
}
% There is something seriously funky happening between the tabular commands and the end of the itemize blocks.
% The command \eatvspace should be used if extra space appears at the end of a datetabular environment.
\end{datetabular}

% \section{Leadership and Teaching Experience}

% \section{Honors}

\section{Publications}
\nocite{*} % Loads every entry from the attached .bib file
% Highlight takes three entries, given name, given name initials, and family name.
% If you have a middle initial, this call looks like:
% \highlightauthorname{Bob H.}{B. H.}{Smith}
\highlightauthorname{Kaiwen}{K.}{Dong} 
\begin{datetabular}{2em}
%printbibyear has been defined to only print entries from a given citation year.
\dateentry{2021}{\printallbib}
\end{datetabular}

\end{document}
